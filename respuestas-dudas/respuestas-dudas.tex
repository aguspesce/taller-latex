\documentclass[a4paper,11pt]{article}
\usepackage[utf8]{inputenc}
\usepackage[spanish]{babel}
\usepackage{braket}
\usepackage{gensymb}
\usepackage[hmargin=3cm, vmargin=3cm]{geometry}
\usepackage{hyperref}
\hypersetup{colorlinks,%
            citecolor=black,%
            filecolor=black,%
            linkcolor=black,%
            urlcolor=blue,%
            pdftex}
%\urlstyle{same}                 % hyperlinks

\title{\textbf{Respuestas a dudas}}
\date{}

\usepackage{lipsum}


\begin{document}

\maketitle

\section{Interlineado}

Para configurar el interlineado en todo el documento es necesario usar el paquete \textbf{setspace} en el preámbulo:
\begin{verbatim}
\usepackage{setspace}
\end{verbatim}

Luego debemos configurar el ancho del interlineado. Para hacerlo de 1.5 debemos agregar la siguiente linea en el preámbulo:
\begin{verbatim}
\onehalfspacing
\end{verbatim}

O bien, si deseamos interlineado doble:
\begin{verbatim}
\onehalfspacing
\end{verbatim}

Para configurar un interlineado arbitrario, utilizamos por ejemplo:
\begin{verbatim}
\setstretch{1.2}
\end{verbatim}


Obtener más información en: \url{https://en.wikibooks.org/wiki/LaTeX/Text_Formatting} y \url{https://en.wikibooks.org/wiki/LaTeX/Paragraph_Formatting#Line_spacing}.


\section{Numeración de lineas}

Para mostrar los números de líneas es necesario usar el paquete \textbf{lineno}. Activamos la numeración con el comando \textbackslash linenumbers.

\begin{verbatim}
\usepackage{lineno}
\linenumbers
\end{verbatim}

Más opciones en \url{http://texblog.org/2012/02/08/adding-line-numbers-to-documents/} y \url{http://texdoc.net/texmf-dist/doc/latex/lineno/ulineno.pdf}.

\section{Notación de Dirac}

Para utilizar la notación de Dirac dentro de los entornos matemáticos es necesario usar el paquete \textbf{brakets}. Insertamos bra, kets y brakets con los comandos \textbackslash Bra \textbackslash Ket y \textbackslash Braket.

\begin{verbatim}
\usepackage{brakets}
\begin{equation}
\Bra{r}\hat{H}\Ket{\varphi} = E\Braket{r | \varphi}
\end{equation}
\end{verbatim}

\begin{equation}
\Bra{r}\hat{H}\Ket{\varphi} =  E\Braket{r | \varphi}
\end{equation}

\section{Símbolos adicionales}

\subsection{Símbolo de grado}

Para utilizar el símbolo de grado necesitamos utilizar el paquete \textbf{gensymb} en el preámbulo:


\begin{verbatim}
\usepackage{gensymb}
\end{verbatim}

Y luego insertamos el símbolo en el cuerpo del texto con el comando \textbf{\textbackslash degree} en modo matemático. Por ejemplo:

\begin{verbatim}
Nuestra zona de estudio se encuentra entre los 30$\degree$S y 34$\degree$S.
En la misma, el gradiente geotérmico responde a un aumento de 30$\degree$C
cada 1km de profundidad.
\end{verbatim}

Nuestra zona de estudio se encuentra entre los 30$\degree$S y 34$\degree$S. En la misma, el gradiente geotérmico responde a un aumento de 30$\degree$C cada 1km de profundidad.


\end{document}
