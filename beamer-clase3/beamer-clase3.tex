\documentclass[11pt]{beamer}
\usetheme[progressbar=frametitle]{metropolis}
%\usetheme{Singapore}
\usepackage[utf8]{inputenc}
\usepackage[spanish]{babel}
\usepackage{fancyvrb}
\definecolor{new_green}{HTML}{347F1F}


\author{Lic. Agustina Pesce \\ Lic. Santiago Soler}
\title[Taller de {\LaTeX}]{Taller Introductorio a {\LaTeX}: \\ Cómo producir documentos de Calidad}
\subtitle{Tercer Encuentro: Bibligrafía}
%\setbeamercovered{transparent} 
%\setbeamertemplate{navigation symbols}{} 
%\logo{} 
%\institute{} 
\date{}
%\subject{} 
\begin{document}

\maketitle

\section{Bibliografía}

\begin{frame}{Bibliografía}
En los artículos científicos es obligación incluir las fuentes a partir de las cuales partimos en el desarrollo del trabajo que queremos dar a conocer. Por un lado esto le da sustento a nuestro trabajo, y por el otro nos permite darle crédito al trabajo de quienes nos preceden, así como nuestros trabajos serán reconocidos cuando sean citados.

\LaTeX{} nos ofrece dos formas de incluir bibliografías (o referencias) y citarlas en el texto:
\begin{itemize}
\item{Bibliografía Simple}
\item{\textsc{Bib}\TeX{}}
\end{itemize}
\end{frame}

\section{Bibliografía Simple}

\begin{frame}[fragile]{Bibliografía Simple}
Si nuestra intención es escribir un documento sencillo, con poca bibliografía que no volveremos a utilizar en un futuro, la opción más recomendada es utilizar el entorno ``thebibliography' al final del documento:

{\color{new_green} \scriptsize
\begin{verbatim}
\renewcommand{\refname}{Bibliografía}
\begin{thebibliography}{2}

\end{thebibliography}
\end{verbatim}
}

Por defecto \LaTeX{} titula la sección de bibliografía como Referencias (como traducción de References en inglés). Para modificarlo hacemos uso de la primer línea.
\end{frame}

\begin{frame}[fragile]{Bibliografía Simple}
Para introducir una entrada en la bibliografía lo hacemos a través de ``\textbackslash bibitem\{label\}''

{\color{new_green} \scriptsize
\begin{verbatim}
\renewcommand{\refname}{Bibliografía}
\begin{thebibliography}{2}

\bibitem{Noether1918}
  Emmy Noether (1918)
  ``Invariante Variationsprobleme''.
  Nachr. D. König. Gesellsch. D. Wiss. Zu Göttingen, Math-phys. Klasse.

\bibitem{Feynman1966}
  Feynman, Richard P. (1966).
  ``The Development of the Space-Time View of Quantum Electrodynamics''.
  Science

\end{thebibliography}
\end{verbatim}
}
\end{frame}


\begin{frame}[fragile]{Bibliografía Simple}
Luego, estas referencias bibliográficas pueden ser citadas en el cuerpo del texto con el comando ``\textbackslash cite\{label\}''.

{\color{new_green} \scriptsize
\begin{verbatim}
Según \cite{Noether1918}, a cada simetría continua de un sistema
físico le corresponde una constante de movimiento, y viceversa.
\end{verbatim}
}

Como etiqueta podemos utilizar lo que deseemos, aunque por lo general se utiliza el apellido del primer autor acompañado por el año de publicación.

\end{frame}

\section{\textsc{Bib}\TeX{}}

\begin{frame}[fragile]{\textsc{Bib}\TeX{}}
La otra opción para incluir bibliografías es hacerlo a través de \textsc{Bib}\TeX{}.

\textsc{Bib}\TeX{} nos permite construir bases de datos de referencias bibliográficas que son almacenadas en archivos ``.bib''.

Veamos un ejemplo de la entrada de un artículo en un .bib:

{\color{new_green} \scriptsize
\begin{verbatim}
@article{greenwade93,
    author  = "George D. Greenwade",
    title   = "The {C}omprehensive {T}ex {A}rchive {N}etwork ({CTAN})",
    year    = "1993",
    journal = "TUGBoat",
    volume  = "14",
    number  = "3",
    pages   = "342--351"
}
\end{verbatim}
}
\end{frame}

\begin{frame}{\textsc{Bib}\TeX{}}
Por suerte, existen herramientas gráficas que nos facilitan la tarea de construir estas bases de datos, incorporando herramientas útiles como búsquedas online.

Una de ellas es JabRef.

\end{frame}



\end{document}