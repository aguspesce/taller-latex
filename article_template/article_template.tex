\documentclass[a4paper]{article}
%\documentclass[a4paper,11pt,twoside,twocoulumn]{article}
\usepackage[spanish]{babel} % diccionario que utilizará el documento, permite acentos y ñ
%\usepackage[english]{babel} % usando inglés, los acentos y ñ se colocan: \'e \~n
\usepackage[utf8]{inputenc} % codificación del texto fuente
\usepackage[T1]{fontenc} % codificación del pdf de salida
\usepackage{lmodern} % latin modern fonts
\usepackage[a4paper,hmargin=2.5cm, vmargin=2.5cm]{geometry} % geometría de la hoja del pdf
\usepackage{graphicx} % soporte mejorado para imágenes
\usepackage{abstract} % permite crear resúmen
\usepackage{amsmath} % entornos adicionales para expresiones matemáticas
\usepackage{amssymb} % símbolos matemáticos adicionales
\usepackage{natbib} % paquete necesario para incluir bibliografía


\title{Template de un Artículo Científico}
\author{Santiago Soler, Agustina Pesce}
\date{\today}

\begin{document}

\maketitle


\begin{abstract}
Este es el resumen del artículo
\end{abstract}



\section{Introducción}
%\section*{Introducción} % si no queremos numerar las secciones agregamos un *

Citar entre paréntesis \citep[e.g.][]{Tanaka1999}.
Citar en texto a \citet{Tanaka1999}.


%~ \bibliographystyle{apalike}
%~ \bibliography{bibliography}
%~ \addcontentsline{toc}{part}{References}
\end{document}
