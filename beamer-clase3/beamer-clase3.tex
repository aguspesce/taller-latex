\documentclass[11pt]{beamer}
\usetheme[progressbar=frametitle]{metropolis}
%\usetheme{Singapore}
\usepackage[utf8]{inputenc}
\usepackage[spanish]{babel}
\usepackage{fancyvrb}
\usepackage{hyperref}
\definecolor{new_green}{HTML}{347F1F}


\author{Lic. Agustina Pesce \\ Lic. Santiago Soler}
\title[Taller de {\LaTeX}]{Taller Introductorio a {\LaTeX}: \\ Cómo producir documentos de Calidad}
\subtitle{Tercer Encuentro: Bibligrafía}
%\setbeamercovered{transparent} 
%\setbeamertemplate{navigation symbols}{} 
%\logo{} 
%\institute{} 
\date{}
%\subject{} 

\newcommand{\BibTeX}{\textsc{Bib}\TeX{}}

\begin{document}

\maketitle


\section{Bibliografía}


\begin{frame}{Bibliografía}
En los artículos científicos es obligación incluir las fuentes a partir de las cuales partimos en el desarrollo del trabajo que queremos dar a conocer. Por un lado esto le da sustento a nuestro trabajo, y por el otro nos permite darle crédito al trabajo de quienes nos preceden, así como nuestros trabajos serán reconocidos cuando sean citados.

\LaTeX{} nos ofrece dos formas de incluir bibliografías (o referencias) y citarlas en el texto:
\begin{itemize}
\item{Bibliografía Simple}
\item{\BibTeX{}}
\end{itemize}
\end{frame}


\section{Bibliografía Simple}


\begin{frame}[fragile]{Bibliografía Simple}
Si nuestra intención es escribir un documento sencillo, con poca bibliografía que no volveremos a utilizar en un futuro, la opción más recomendada es utilizar el entorno ``thebibliography' al final del documento:

{\color{new_green} \scriptsize
\begin{verbatim}
\renewcommand{\refname}{Bibliografía}
\begin{thebibliography}{2}

\end{thebibliography}
\end{verbatim}
}

Por defecto \LaTeX{} titula la sección de bibliografía como Referencias (como traducción de References en inglés). Para modificarlo hacemos uso de la primer línea.
\end{frame}


\begin{frame}[fragile]{Bibliografía Simple}
Para introducir una entrada en la bibliografía lo hacemos a través de ``\textbackslash bibitem\{label\}''

{\color{new_green} \scriptsize
\begin{verbatim}
\renewcommand{\refname}{Bibliografía}
\begin{thebibliography}{2}

\bibitem{Noether1918}
  Emmy Noether (1918)
  ``Invariante Variationsprobleme''.
  Nachr. D. König. Gesellsch. D. Wiss. Zu Göttingen, Math-phys. Klasse.

\bibitem{Feynman1966}
  Feynman, Richard P. (1966).
  ``The Development of the Space-Time View of Quantum Electrodynamics''.
  Science

\end{thebibliography}
\end{verbatim}
}
\end{frame}


\begin{frame}[fragile]{Bibliografía Simple}
Luego, estas referencias bibliográficas pueden ser citadas en el cuerpo del texto con el comando ``\textbackslash cite\{label\}''.

{\color{new_green} \scriptsize
\begin{verbatim}
Según \cite{Noether1918}, a cada simetría continua de un sistema
físico le corresponde una constante de movimiento, y viceversa.
\end{verbatim}
}

Como etiqueta podemos utilizar lo que deseemos, aunque por lo general se utiliza el apellido del primer autor acompañado por el año de publicación.

\end{frame}


\section{\BibTeX{}}


\begin{frame}[fragile]{\BibTeX{}}
La otra opción para incluir bibliografías es hacerlo a través de \BibTeX{}.

\BibTeX{} nos permite construir bases de datos de referencias bibliográficas que son almacenadas en archivos ``.bib''.

Veamos un ejemplo de la entrada de un artículo en un .bib:

{\color{new_green} \scriptsize
\begin{verbatim}
@article{greenwade93,
    author  = "George D. Greenwade",
    title   = "The {C}omprehensive {T}ex {A}rchive {N}etwork ({CTAN})",
    year    = "1993",
    journal = "TUGBoat",
    volume  = "14",
    number  = "3",
    pages   = "342--351"
}
\end{verbatim}
}
\end{frame}


\begin{frame}{\BibTeX{} - JabRef}
Por suerte, existen herramientas gráficas que nos facilitan la tarea de construir estas bases de datos, incorporando herramientas útiles como búsquedas online.

Una de ellas es JabRef.

\end{frame}


\begin{frame}[fragile]{\BibTeX{} - Inserción en Documento}
Podemos insertar la base de datos .bib muy sencillamente en nuestro documento. Para ello agregamos lo siguiente en el lugar donde queremos posicionar la Bibliografía:

{\color{new_green} \scriptsize
\begin{verbatim}
\bibliographystyle{plain}
\bibliography{base-de-datos.bib}
\end{verbatim}
}

En la primer línea establecemos el formato con el cual se generará la bibliografía, mientras que en la segunda importamos el archivo .bib.
\end{frame}

\begin{frame}[fragile]{\BibTeX{} - Citas}
Al igual que con la bibliografía simple, podemos citar entradas de la siguiente manera:

{\color{new_green} \scriptsize
\begin{verbatim}
\cite{Apellido2017}
\end{verbatim}
}

A diferencia de la bibliografía simple, en la lista de fuentes bibliográficas generadas con \BibTeX{} solo aparecerán aquellas que son citadas en el texto. Si deseamos incluir alguna entrada que no citamos, lo podemos hacer de la siguiente manera:

{\color{new_green} \scriptsize
\begin{verbatim}
\nocite{Apellido2005}
\end{verbatim}
}

O bien, si queremos incluir toda la base de datos:

{\color{new_green} \scriptsize
\begin{verbatim}
\nocite{*}
\end{verbatim}
}
\end{frame}


\begin{frame}[fragile]{\BibTeX{} - Estilos de Bibliografía}

Existen varios estilos de bibliografía:

\begin{columns}
\column{0.25\textwidth}
\begin{itemize}
\item plain
\item apalike
\end{itemize}
\column{0.25\textwidth}
\begin{itemize}
\item abbrv
\item acm
\end{itemize}
\column{0.25\textwidth}
\begin{itemize}
\item alpha
\item ieeetr
\end{itemize}
\column{0.25\textwidth}
\begin{itemize}
\item siam
\item unsrt
\end{itemize}
\end{columns}

\vspace{2em}

{\footnotesize \url{https://www.sharelatex.com/learn/Bibtex_bibliography_styles}}

\end{frame}


\begin{frame}[fragile]{\BibTeX{} - Natbib}
Cuando utilizamos \BibTeX{} es muy recomendable importar el paquete ``natbib''.

{\color{new_green} \scriptsize
\begin{verbatim}
\usepackage{natbib}
\end{verbatim}
}

Natbib modifica el esquema de citas. En vez de introducir la cita entre corchetes, lo hace limpiamente o entre paréntesis. Además incorpora características estéticas, como por ejemplo abreviar con et.al. los nombres en publicaciones con más de dos autores.

El comando ``\textbackslash cite{}'' se divide en dos nuevos comandos:
{\color{new_green} \scriptsize
\begin{verbatim}
\citet{} % para citas en linea
\citep{} % para citas entre paréntesis
\end{verbatim}
}

\end{frame}


\begin{frame}[fragile]{\BibTeX{} - Natbib}

{\footnotesize
\begin{tabular}{l | l}
\textbf{Citation command} 	& \textbf{Output} \\ \hline
\textbackslash citet\{goossens93\}       & Goossens et al. (1993)                    \\
\textbackslash citep\{goossens93\}       & (Goossens et al., 1993)                   \\ \hline
\textbackslash citet*\{goossens93\}      & Goossens, Mittlebach, and Samarin (1993)  \\
\textbackslash citep*\{goossens93\}      & (Goossens, Mittlebach, and Samarin, 1993) \\ \hline
\textbackslash citeauthor\{goossens93\}  & Goossens et al.                           \\
\textbackslash citeauthor*\{goossens93\} & Goossens, Mittlebach, and Samarin         \\ \hline
\textbackslash citeyear\{goossens93\}    & 1993                                      \\
\textbackslash citeyearpar\{goossens93\} & (1993)                                    \\ \hline
\textbackslash citealt\{goossens93\}     & Goossens et al. 1993                      \\
\textbackslash citealp\{goossens93\}     & Goossens et al., 1993                     \\
\end{tabular}
}

\vspace{1em}

{\small Podemos agrupar citas entre paréntesis:}
\vspace{-1em}

{\color{new_green} \scriptsize
\begin{verbatim}
\citep{AutorA1998, AutorB2003, AutorC2005}
\end{verbatim}
}

{\small Y podemos agregar textos dentro de los paréntesis:}
\vspace{-1em}

{\color{new_green} \scriptsize
\begin{verbatim}
\citep[e.g.][otros]{AutorA1998, AutorB2003, AutorC2005}
\end{verbatim}
}

\end{frame}

\end{document}
