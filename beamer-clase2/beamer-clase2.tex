\documentclass[11pt]{beamer}
\usetheme[progressbar=frametitle]{metropolis}
\usepackage[utf8]{inputenc}
\usepackage[spanish]{babel}
\usepackage{fancyvrb}
\definecolor{new_green}{HTML}{347F1F}


\author{Lic. Agustina Pesce \\ Lic. Santiago Soler}
\title[Taller de {\LaTeX}]{Taller Introductorio a {\LaTeX}: \\ Cómo producir documentos de Calidad}
\subtitle{Segundo Encuentro: Artículo Científico}
%\setbeamercovered{transparent} 
%\setbeamertemplate{navigation symbols}{} 
%\logo{} 
%\institute{} 
\date{}
%\subject{} 

\begin{document}
\maketitle

\section{El Articulo Científico}

\begin{frame}{Estructura Básica de un Artículo Científico}
 \begin{itemize}[<+- | alert@+>] % Con esto hacemos que aparezcan los items de a uno
  \item Título
  \item Autores
  \item Abstract (resumen)
  \item Cuerpo (secciones y subsecciones)
 \end{itemize}
\end{frame}

\section{¿Cómo se hace en \LaTeX{}?}

\begin{frame}[fragile]{Artículo Científico}
 \textbf{Ejemplos de un articulo en \LaTeX{}}
 \begin{columns}
  \column{0.6\textwidth}
  {\color{new_green}
  \begin{Verbatim}[fontsize=\scriptsize]
   \documentclass[a4paper,12pt]{article}
   \usepackage[utf8]{inputenc}
   
   \title{Acá coloco el título del documento}
   \author{El nombre del/los autor/es}
   
   \begin{document}
   \maketitle
   
   \begin{abstract}
    Con este entorno creo
    el resumen del trabajo.
   \end{abstract}

   \section{Nombre de la 1ra Sección}
   Acá empiezo a escribir mi trabajo.
   \subsection{Nombre de la Subsección}
   \end{document}
  \end{Verbatim}
  }
   \hfill
   \column{0.4\textwidth}
   {\scriptsize
   Declaro el tipo de documento, papel y tamaño de letra. 
   Defino la codificación del documento. \\
   \vspace{2.5em}
   Defino el título. \\
   Defino el autor. \\
   \vspace{2.5em}
   Creo el título y el autor \\
   \vspace{2.5em}
   Creo el Abstract \\
   \vspace{2em}
   Creo una sección y una subsección.
   Usando $*$ no se numera la sección.
   \vspace{4.5em}
   }
 \end{columns}
\end{frame}


\begin{frame}{Artículo Científico}
 \begin{center}
 Comencemos a escribir nuestro primer artículo en \LaTeX{}...
 \end{center}
\end{frame}

\begin{frame}{Artículo Científico}
 \begin{center}
  ¿Qué más necesito para hacer un artículo?
 \end{center}
\pause
 \begin{itemize}[<+- | alert@+>] % Con esto hacemos que aparezcan los items de a uno
  \item Insertar ecuaciones
  \item Insertar figuras
  \item Insertar tablas
  \item Hacer referencias cruzadas
  \item Notas de pié de página
 \end{itemize}
\end{frame}

\begin{frame}[fragile]{Artículo Científico - Ecuaciones}
 \begin{columns}
  \column{0.7\textwidth}
  {\color{new_green}
  \begin{Verbatim}[fontsize=\scriptsize]
   \begin{equation}
   A_{m,n} = 
  
  \begin{pmatrix}
   a_{1,1} & a_{1,2} & \cdots & a_{1,n} \\
   a_{2,1} & a_{2,2} & \cdots & a_{2,n} \\
   \vdots  & \vdots  & \ddots & \vdots  \\
   a_{m,1} & a_{m,2} & \cdots & a_{m,n} 
  \end{pmatrix}
  
  \label{Nombre para la referencia curzada.} 
  \end{equation}
  \end{Verbatim}
  }
 \end{columns}
 \begin{equation}
  {\scriptsize
  A_{m,n} = 
  \begin{pmatrix}
   a_{1,1} & a_{1,2} & \cdots & a_{1,n} \\
   a_{2,1} & a_{2,2} & \cdots & a_{2,n} \\
   \vdots  & \vdots  & \ddots & \vdots  \\
   a_{m,1} & a_{m,2} & \cdots & a_{m,n} 
  \end{pmatrix}
  \label{Nombre para la referencia curzada.} 
  }
 \end{equation}
\end{frame}

\begin{frame}[fragile]{Artículo Científico - Referencias Cruzadas}
Con \LaTeX{}, podemos hacer fácilmente referencias a figuras, tablas y ecuaciones de nuestro documento.
El mecanismo es muy fácil:
$\backslash label \{ X \}$ - Con este comando se etiqueta el objeto. \\
$\backslash ref \{ X \}$ - Con este comando se referencia  el objeto en el texto.
 
La etiqueta puede ser elegida libremente sin ningún criterio, aunque es recomendable agregar un pequeño indicador de qué objeto estamos referenciando. Por ejemplo, en el caso de ecuaciones:
 
\begin{columns}
\column{0.4\textwidth}
{\color{new_green} \scriptsize
\begin{verbatim}
\begin{equation}
  {d\over dt }\frac{\partial L} -
  \frac{\partial L}{\partial x} 
  {\partial \dot{x}} = 0
\label{eq:lagrange}
\end{equation}
\end{verbatim}
}
\column{0.4\textwidth}
\begin{equation}
  \frac{\partial L}{\partial x} - 
  {d\over dt }\frac{\partial L}
  {\partial \dot{x}} = 0
\label{eq:lagrange}
\end{equation}
\end{columns}

\end{frame}

\begin{frame}[fragile]{Artículo Científico - Figuras}
 Se necesita agregar en el preámbulo el paquete:
 \textbackslash usepackage\{graphicx\}
 \vspace{1em}
 \begin{columns}
  \column{0.8\textwidth}
  {\color{new_green}
  \begin{Verbatim}[fontsize=\scriptsize]
   \begin{figure}[Posición]
    \centering
    \includegraphics[width=0.5\textwidth]{figure/name.png}
    \caption{Epígrafe.}
    \label{Nombre para la referencia curzada.}
   \end{figure}
  \end{Verbatim}
  }
  \hfill
  \column{0.3\textwidth}
  \begin{table}
   \scriptsize
    \begin{tabular}[h]{|c|l|}
     \hline
     \multicolumn{2}{|c|}{Posición} \\
     \hline
     h & Acá \\ \hline
     t & Top \\ \hline
     b & Bottom \\ \hline
     p & New page \\ \hline
     ! & Forzado \\ \hline
    \end{tabular}
  \end{table}
 \end{columns}
\end{frame}

\begin{frame}[fragile]{Artículo Científico - Figuras}
Al igual que con las ecuaciones, utilizamos una etiqueta en particular para las referencias de figuras. Por ejemplo:
  {\color{new_green}
  \begin{Verbatim}[fontsize=\scriptsize]
   \begin{figure}[Posición]
    \centering
    \includegraphics[width=0.5\textwidth]{figure/name.png}
    \caption{Epígrafe.}
    \label{fig:ejemplo-de-figura}
   \end{figure}
  \end{Verbatim}
  }

\end{frame}

\begin{frame}{Artículo Científico - Figuras}
 \begin{center}
  Su turno de hacerlo... 
 \end{center}
\end{frame}

\begin{frame}[fragile]{Artículo Científico - Tablas}
 El entorno  \textbf{table y tabular} se puede utilizar para incertar tablas con líneas horizontales y verticales opcionales.
 \LaTeX{} determina automáticamente el ancho de las columnas.
 {\color{new_green}
 \begin{Verbatim}[fontsize=\scriptsize]
  \begin{table}[Posición]
   \begin{tabular}{Aspecto}
    Acá escribo la tabla...
   \end{tabular}
   \caption{Epígrafe.}
   \label{Nombre de la referencia cruzada}
  \end{table}
 \end{Verbatim}
 }
 \begin{table}
  \scriptsize
  \begin{tabular}[h]{|c|l|c|l|c|l|}
   \hline
   \multicolumn{2}{|c|}{Posición} & \multicolumn{2}{|c|}{Aspecto} & \multicolumn{2}{|c|}{Otros}\\
   \hline
   h & Acá      & l    & Justificado izquierdo & \&                      & Separador de columnas \\ \hline
   t & Top      & c    & Centrado              & $\backslash \backslash$ & Iniciar nueva linea\\ \hline
   b & Bottom   & r    & Justificado derecho   & $\backslash$hline       & Linea horizontal \\ \hline
   p & New page &  $|$ & Linea vertical        &                         &    \\ \hline
   ! & Forzar   & $\|$ & Linea vertical doble  &                         &     \\ \hline
  \end{tabular}
  \end{table}
\end{frame}

\begin{frame}[fragile]{Artículo Científico - Tablas}
 Ejemplo:
 \begin{columns}
  \column{0.6\textwidth}
  {\color{new_green}
  \begin{Verbatim}[fontsize=\scriptsize]
   \begin{table}
    \begin{tabular}{ll|r}
     \hline
     \multicolumn{2}{c|}{Item} \\
     \cline{1-2}
     Animal    & Description & Price (\$) \\
     \hline
     Gnat      & per gram    & 13.65      \\
               & each        & 0.01       \\
     Gnu       & stuffed     & 92.50      \\
     Emu       & stuffed     & 33.33      \\
     \hline
    \end{tabular}
   \end{table}
  \end{Verbatim}
  }
  \hfill
  \column{0.5\textwidth}
  \begin{table}
   \scriptsize
   \begin{tabular}{ll|r}
    \hline
    \multicolumn{2}{c|}{Item} \\
    \cline{1-2}
    Animal    & Description & Price (\$) \\
    \hline
    Gnat      & per gram    & 13.65      \\
              & each        & 0.01       \\
    Gnu       & stuffed     & 92.50      \\
    Emu       & stuffed     & 33.33      \\
    \hline
   \end{tabular}
  \end{table}
 \end{columns}
\end{frame}

\begin{frame}[fragile]{Artículo Científico - Nota de pié de página}
 Las notas de pié de página deben colocarse siempre después de la palabra o frase a la que se refieren.
 La nota se imprime al pie de la página actual. 
 {\color{new_green}
   \begin{Verbatim}[fontsize=\scriptsize]
    Las notas de pié de pégina\footnote{Esto es una nota de pié de página.}
    son muy usadas por las personas que usan \LaTeX.
   \end{Verbatim}
  }
  Las notas de pié de página\footnote{Esto es una nota de pié de página.}
  son muy usadas por las personas que usan \LaTeX.
\end{frame}


\end{document}
