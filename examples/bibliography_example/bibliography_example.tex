\documentclass[12pt,a4paper]{article}
\usepackage[utf8]{inputenc}
\usepackage[spanish]{babel}
\usepackage[left=2cm,right=2cm,top=2cm,bottom=2cm]{geometry}
\author{Autor}
\title{Bibliografía Simple}

\begin{document}

\maketitle

Según \cite{Noether1918}, a cada simetría continua de un sistema físico le corresponde una constante de movimiento, y viceversa.

\renewcommand{\refname}{Bibliografía} % cambiamos el nombre de la bibliografia, de Referencias a Bibliografía
\begin{thebibliography}{2}
\bibitem{Noether1918} Emmy Noether (1918) ``Invariante Variationsprobleme''. Nachr. D. König. Gesellsch. D. Wiss. Zu Göttingen, Math-phys. Klasse.
\bibitem{Feynman1966} Feynman, Richard P. (1966). ``The Development of the Space-Time View of Quantum Electrodynamics''. Science
\end{thebibliography}

\end{document}